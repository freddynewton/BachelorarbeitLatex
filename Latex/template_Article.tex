 \documentclass[]{report}

%\usepackage{harvard}

\usepackage{float}
\usepackage{amsmath}
\usepackage{amsfonts}
\usepackage{graphicx}
\usepackage{hyperref}
\usepackage[authoryear,square]{natbib}
\usepackage[svgnames]{xcolor}
\usepackage[strings]{underscore}
\usepackage[printonlyused]{acronym}
\bibliographystyle{plainnat}



%opening
\title{Tactical Game AI with Shared Knowledge based on Influence Maps}
\author{Fred Newton, Akdogan}

\begin{document}
	
	%----------------------------------------------------------------------------------------
	%	TITLE PAGE
	%----------------------------------------------------------------------------------------
	
	\begin{titlepage}
		\centering
		
		%------------------------------------------------
		%	Top rules
		%------------------------------------------------
		
		\rule{\textwidth}{1pt} % Thick horizontal rule
		
		\vspace{2pt}\vspace{-\baselineskip} % Whitespace between rules
		
		\rule{\textwidth}{0.4pt} % Thin horizontal rule
		{\small \text{Bachelor thesis in the Computer Science and Media degree programme}}
		\vspace{0.1\textheight} % Whitespace between the top rules and title
		
		%------------------------------------------------
		%	Title
		%------------------------------------------------
		
		
		\textcolor{black}{ % Red font color
			{\Huge Tactical Game AI}\\[0.5\baselineskip] % Title line 1
			{\Huge with shared Knowledge}\\[0.5\baselineskip] % Title line 2
			{\Huge based on Influence Maps} % Title line 3
		}
		
		\vspace{0.025\textheight}
		
		\rule{0.8\textwidth}{0.4pt}
		
		\vspace{0.1\textheight}
		
		%------------------------------------------------
		%	Author
		%------------------------------------------------
		
		\textcolor{black}{ % Red font color
			{\small submitted by}\\[0.5\baselineskip] % Title line 1
			{\Large \textsc{Fred Newton, Akdogan}}\\[0.5\baselineskip] % Title line 2
		}
		
		\vspace{0.025\textheight} % Whitespace between the title and short horizontal rule
		
		{\small at the Stuttgart Media University on August 30, 2021\\[0.5\baselineskip]}
		{\small to obtain the academic degree of Bachelor of Science (B.Sc)\\[0.5\baselineskip]}
		
		\vfill % Whitespace between the author name and publisher
		
		%------------------------------------------------
		%	Publisher
		%------------------------------------------------
		
		%{\large\textcolor{Red}{\plogo}}\\[0.5\baselineskip] % Publisher logo
		
		{\small First examiner: Prof. Dr. Stefan Radicke\\[0.5\baselineskip]}
		{\small Second examiner: Prof. Dr. Joachim Charzinski\\[0.5\baselineskip]}
		
		\vspace{0.1\textheight} % Whitespace under the publisher text
		
		%------------------------------------------------
		%	Bottom rules
		%------------------------------------------------
		
		\rule{\textwidth}{0.4pt} % Thin horizontal rule
		
		\vspace{2pt}\vspace{-\baselineskip} % Whitespace between rules
		
		\rule{\textwidth}{1pt} % Thick horizontal rule
		
	\end{titlepage}
	
	
	\begin{abstract}
		
	\end{abstract}
	
	\newpage
	\chapter*{Honorary Declaration}
	\thispagestyle{empty}
	Hiermit versichere ich, Fred Newton Akdogan, ehrenwörtlich, dass ich die
	vorliegende Bachelorarbeit (bzw. Masterarbeit) mit dem Titel: "Tactical Game AI with Shared Knowledge based on Influence Maps" selbstständig und ohne fremde Hilfe verfasst und keine
	anderen als die angegebenen Hilfsmittel benutzt habe. Die Stellen der Arbeit, die dem
	Wortlaut oder dem Sinn nach anderen Werken entnommen wurden, sind in jedem Fall
	unter Angabe der Quelle kenntlich gemacht. Die Arbeit ist noch nicht veröffentlicht oder
	in anderer Form als Prüfungsleistung vorgelegt worden.
	Ich habe die Bedeutung der ehrenwörtlichen Versicherung und die prüfungsrechtlichen
	Folgen (§26 Abs. 2 Bachelor-SPO (6 Semester), § 24 Abs. 2 Bachelor-SPO (7 Semester), §
	23 Abs. 2 Master-SPO (3 Semester) bzw. § 19 Abs. 2 Master-SPO (4 Semester und
	berufsbegleitend) der HdM) einer unrichtigen oder unvollständigen ehrenwörtlichen
	Versicherung zur Kenntnis genommen.
	
	\vfill
	\noindent\begin{tabular}{ll}
		\makebox[2.5in]{\hrulefill} & \makebox[2.5in]{\hrulefill}\\
		Signature & Date\\[8ex]
	\end{tabular}
	\newpage
	
	\newpage
	\pagenumbering{arabic} 
	\tableofcontents

	\chapter{Introduction}
	% TODO Write about about what it is this thesis
	
	\section{Motivation}
During his studies, Mr Akdogan always wondered how the game AI shares its knowledge. Because if the \ac{AI} always knows everything, it would not be beneficial for the player and you want to give the player a good experience. While looking for a topic for his bachelor thesis, he came across a \ac{GDC} video from \citep{knowledgeReprentation} that talks about the representation of information and also about influence maps for the \ac{AI}. This raised the question of how much influence \ac{SK} has among \ac{AI} as opposed to agents not sharing their knowledge with each other.
	
	\section{Scientific question}
	How big is the difference between the \ac{AI} agents when they share their influence map or each agent uses its own influence map? 
	\section{Structure of the Thesis}
	% TODO
	
	\chapter{Related work}
	In \citep{gameDevInfluenceMap} article discussed how \ac{IM} works in general. As well as the important part of giving an agent or squad a memory of the current influence range on the map. 
	
	\chapter{Theoretical background}
	\section{Influence Map}
	All information is taken from the book \ac{AI} for Games unless otherwise cited in this chapter \citep{AIforGamesTactical}.
	
	An \ac{IM} is used to record the current balance of Military Influence at each position in a level. Many factors can affect military influence, such as the proximity of a unit, the proximity of a base, the length of time a unit has been last seen, the terrain, the current weather, the strength of a unit. Many factors have only a small influence. Based on how the abstract image is drawn across the map or level of influence, more marginal information can be communicated to the \ac{AI} and tactical decisions can be made. 
	
	\subsection{Simple Influence} \label{ssec:num2}
	The influence of a unit in an area consists of how much its influence is weighted. Assuming it is a Real Time Strategy game and there is a foot soldier unit and a tank. Normally, a tank has more lives, more damage and a longer range than a simple foot soldier. This means that a larger Influence value is taken and injected into the \ac{IM} at the unit's position. If you take the strength of a unit, it decreases with increasing distance. So the further away you are from the unit, the less influence it has. A linear drop-off model can be used for this. A doubling of the distance results in a halving impact:
	
	\begin{equation}
		I_{d} = \frac{I_{0}}{1 + d}
	\end{equation}

	
	$I_{d}$ is the influence at a given distance. $d$ is the distance from the unit to the point and $I_{0}$ is the influence at the distance value 0 to the unit. It would also be possible to use a more sloping initial drop off, with a greater range of influence:
	
	\begin{equation}
		I_{d} = \frac{I_{0}}{\sqrt{1 + d}}
	\end{equation}
	
	It is also possible to use an equation that first flattens out and then falls sharply: 
	
	\begin{equation}
		I_{d} = \frac{I_{0}}{(1 + d)^2}
	\end{equation}
	
	\subsection{Calculating the Influence}
	For the \ac{IM}, a large calculation is needed for each unit on the map for each possible position. The execution time would be $O(nm)$ and the memory is $O(m)$. $m$ represent the number of possible positions in the game and $n$ the number of units. With a linear drop-off curve, the influence is covered with a threshold value. In this way, small values are not unnecessarily stacked on top of each other in a larger range:
	
	\begin{equation}
		r = \frac{I_0}{I_t - 1}
	\end{equation}

	Where $I_t$ is the threshold value for the influence. Thus, the influence of each unit is only applied to the places that are within the given radius. This limits the calculation time to $O(nr)$ for the time and to $O(m)$ for the memory. $r$ is the number of locations that are within the given radius.
	
	\subsection{Dealing with unknowns}
	Here, only the influence of units that can be seen in their radius is calculated for the unit. Thus, an aspect called \ac{FOW} is built in. This is important for investigating whether the \ac{SK} of units makes a difference. In this way, units also have a maximum distance they can see and can only build a personal \ac{IM} based on the friendly or enemy units they can see and incorporate this into their decisions. This can lead to problems for the \ac{AI} decision making, because it does not have the same memory as humans have and cannot map the context. Therefore, it is important to give the \ac{AI} some kind of memory. This can be mapped well with IM so that the \ac{AI} can manage well in the \ac{FOW}. It will also be very interesting to see if it makes a significant difference between the Shared and Unshared Knowledge Teams.
	
	\subsection{Influence Map Setup}
	All information in this section is quoted from the article \citep{gameDevInfluenceMap}. Otherwise they are cited as such.
	
	For the \ac{IM}, a 2-dimensional grid is stretched over the map and divided into a grid system. Then, all areas that cannot be walked on, such as walls or similar, are excluded from the calculation and ignored. This enables the unknown and the FOW. Because with this, the influence does not propar through obstacles but around them. After that, the cells that have the shortest distance to each agent are injected with their influence in the 2 dimensional grid. This means that these cells are always set to the influence value of the unit regardless of anything else. After setting the values of each agent in the IM, a blur algorithm is applied as explained in the chapter \ref{ssec:num1}. The choice of which blur method is best suited for each game depends heavily on the game designer. For example, depending on whether there are many walls on the level or more open vials, it can be decided whether to use one of the distance-based blur algorithms described in Chapter \ref{ssec:num2} or a blur algorithm from Chapter \ref{ssec:num1}. The important thing is to only apply this to cells that are accessible. Thus, the influence must propargize around walls and no distorted image is transmitted. The value from the blur algorithm is then multiplied by the decay to implement a decay of the influence on the range. 
	
	\begin{equation}
		I_{xy} = b_{xy} * D
	\end{equation}
	
	$I_{xy}$ is the influence at the point $x$ and $y$ in the grid. This is equal to the blurred value $b_{xy}$ at the point $x$ and $y$ from the algorithm multiplied by the decay value $D$.
	
	
	\begin{description}
		\item[$\bullet$ Momentum] With momentum the influence is linearly interpolated from the cell. In this way the memory of the IM is suggested.
		\begin{equation}
			I_{xy} = I_{xy} TODO
		\end{equation}
		\item[$\bullet$ Decay] Decay is for the decay of the influence value within an \ac{IM} so a kind of fading memory is built up and the influence continues to decrease depending on how far it is from its point of origin.
		\item[$\bullet$ Update Frequency] This parameter describes how often the influence is updated. 
		\end {description}
		
		\section{Blur } \label{ssec:num1}
		For the calculation of the influence on maps with narrow corridors and small areas a blur algorithm is used \citep{gameDevInfluenceMap}. This suggests that the algorithm is not affected by obstacles. Thus the influence flows around the corners. In this work, a boxblur algorithm was applied. But it would work just as well with a Gaussian blur. This is open to the individual preferences of how the influence should spread. For this work, the boxblur algorithm was chosen because it provides a more uniform distribution in all directions of influence. 
		
		\begin{table}[H]
			\centering
			\begin{tabular}{|c|c|c|c|c|}
				\hline
				0.000 & 0.000 & 0.000 & 0.000 & 0.000\\
				\hline
				0.000 & \textbf{1.000} & 0.000 & 0.000 & 0.000\\
				\hline
				0.000 & 0.000 & 0.000 & 0.000 & 0.000\\
				\hline
				0.000 & 0.000 & 0.000 & 0.000 & 0.000\\
				\hline
				0.000 & 0.000 & \textbf{-1.000} & 0.000 & 0.000\\
				\hline
				
			\end{tabular}
			\caption{Boxblur influence example grid - Iteration 0. The inflow of two units was injected into a completely new \ac{IM}. The cell with the value $1.000$ is from the own team and the cell with the value $-1.000$ from the opposing team. }
			\label{tab:Boxblur grid Iteration 0}
		\end{table}
		
		The Box Blur is a spatial domain linear filter. It takes a pixel (or in our case a cell) from the grid and takes itself and its surrounding pixels and calculates the average as the new value. A 3 by 3 box blur (radius 1) can also be described as a matrix:
		
		\begin{equation}
			K = \frac{1}{9}\begin{bmatrix} 1 & 1 & 1\\ 1 & 1 & 1\\ 1 & 1 & 1 \end{bmatrix}
		\end{equation}
	
		$K$ is the average value of pixel and its surrounding values.\citep{boxblur}
		
		\begin{table}[H]
			\centering
			\begin{tabular}{|c|c|c|c|c|}
				\hline
				0.250 & 0.200 & 0.200 & 0.030 & 0.008\\
				\hline
				0.240 & \textbf{1.000} & 0.160 & 0.006 & 0.007\\
				\hline
				0.210 & 0.180 & 0.150 & 0.040 & 0.002\\
				\hline
				0.060 & -0.040 & -0.070 & -0.100 & -0.010\\
				\hline
				0.005 & -0.170 & \textbf{-1.000} & -0.200 & -0.078\\
				\hline
				
			\end{tabular}
			\caption{Boxblur influence example grid - Iteration 1. After the first iteration, from top left to bottom right, the cell was placed the 3 by 3 matrix. So that the center of the matrix lies on the cell and then calculated the mean value.}
			\label{tab:Boxblur grid Iteration 1}
		\end{table}
		
		\begin{table}[H]
			\centering
			\begin{tabular}{|c|c|c|c|c|}
				\hline
				0.420 & 0.370 & 0.290 & 0.080 & 0.025\\
				\hline
				0.400 & \textbf{1.000} & 0.250 & 0.009 & 0.027\\
				\hline
				0.300 & 0.250 & 0.140 & 0.030 & -0.007\\
				\hline
				0.068 & -0.057 & -0.130 & -0.15 & -0.069\\
				\hline
				-0.039 & -0.221 & \textbf{-1.000} & -0.271 & -0.142\\
				\hline
				
			\end{tabular}
			\caption{Boxblur influence example grid - Iteration 2}
			\label{tab:Boxblur grid Iteration 2}
		\end{table}
		
		Through this box blur, the influence is gradually expanded with $1.000$ and $-1.000$ and then stagnates after a certain iteration. The Celle with $1.000$ means that there is an allied unit that places its influence there and $-1.000$ means that there is an enemy unit.
		
		\section{Waypoints}
		Waypoints are positions distributed around the game world. With waypoints, the \ac{AI} can use this for its pathfinder in order to progress in the game world. Tactical waypoints require more data describing these points in order to make a correct decision about which waypoint to use \citep{AIforGamesTactical}.
		
		\chapter{Experimental setup}
		The experiment is a match against two sides in the style of "conquest" as in games of Battlefield. In Conquest, there are a certain number of places that a side tries to capture. At the beginning, each place is still uncaptured. When a team has taken a point, it always gets points added to their points account at certain time intervals. The team that has reached a certain number of points first after a certain time has won. 
		One side will consist of a squad of five agents, the other side of five squads of one agent each. A squad always knows where its agents are and whether a unit sees a friendly or enemy agent and can thus build up an \ac{IM}.
		Each time a team wins, this is saved in a file and the next match starts from the beginning. This allows you to run this several thousand times so that the result is not falsified.
		
		\section{Assumption}
		I think the side with the squad and its five agents has a slightly higher win rate than the side with the 5 squads with one agent each. This is because it has more information to decide, for example, which of the points to attack first or which point may have no enemy units. 
		
		\section{Rules of the game}
		The goal of one side is to reach 100 points to win the game. This means that there will always be 3 points distributed on the map, each of which can be captured. A point is captured when an agent is in a certain area without any other enemy agents. Each point adds one point to the score every 5 seconds. Each agent has the option to shoot and kill an enemy agent. Each agent has 100 lives and can shoot a projectile with 10 damage every 3 seconds. If the projectile hits a wall or another enemy agent it is destroyed. If an agent has 0 life it is destroyed and after 10 seconds it is re-instantiated at a random location on the map but not in a capturable point and thus rejoins the game.
		
		\subsection{Playing field construction}
		\subsection{Randomness factor}
		
		\chapter{Results}
		\chapter{Discussion}
		
		\chapter{Conclusion}
		
		
		\newpage
		\chapter{Lists}
		\section*{List of Acronyms}
		\begin{acronym}[FOW]
			\acro{IM}{influence map}
			\acro{FOW}{fog-of-war}
			\acro{AI}{artificial intelligence}
			\acro{GDC}{game developers conference}
			\acro{SK}{shared knowledge}
		\end{acronym}
		
		\listoffigures
		\listoftables
		
		
		\bibliography{references}
	\end{document}
